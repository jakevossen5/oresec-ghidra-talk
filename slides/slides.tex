\documentclass{beamer}
%
% Choose how your presentation looks.
%
% For more themes, color themes and font themes, see:
% http://deic.uab.es/~iblanes/beamer_gallery/index_by_theme.html
%

% Theme License: https://creativecommons.org/licenses/by/4.0/

\mode<presentation>
{
  \usetheme{metropolis}   % https://github.com/matze/mtheme
  \usecolortheme{default} % or try albatross, beaver, crane, ...
  \usefonttheme{default}  % or try serif, structurebold, ...
  \setbeamertemplate{navigation symbols}{}
  \setbeamertemplate{caption}[numbered]
} 

\usepackage[english]{babel}
\usepackage[utf8x]{inputenc}

\title[Ghidra]{An Introduction to Reverse Engineering}
\author{Jake Vossen}
\institute{Colorado School of Mines - oresec}
\date{2019-03-19}

\begin{document}

\begin{frame}
  \titlepage
\end{frame}

% Uncomment these lines for an automatically generated outline.
%\begin{frame}{Outline}
%  \tableofcontents
%\end{frame}

\section{Introduction}

\begin{frame}{What is Software Reverse Engineering?}

\begin{itemize}
  \item IEEE defines it as ``he process of analyzing a subject system to identify the system's components and their interrelationships and to create representations of the system in another form or at a higher level of abstraction''
  \item Generally is taking a piece of compiled software and analyzing
    it, revealing information about the source code
  \item Often used in security research, but also have implication in
    game emulation and other areas of proprietary software
  \item Also used to analyze malware to create figure out how to get
    around ransomware and other attacks
\end{itemize}

\vskip 1cm

\end{frame}

\begin{frame}{Why Ghidra?}
  \begin{figure}
    \includegraphics[width=\textwidth]{hex-rays-pricing.png}
    %% \caption{\label{fig:your-figure}Hex Rays Pricing}
  \end{figure}
\end{frame}

\begin{frame}{And if that wasn't enough...}
  \begin{figure}
    \includegraphics[width=\textwidth]{ida-shipping.png}
    %% \caption{\label{fig:your-figure}Hex Rays Pricing}
  \end{figure}
\end{frame}

\begin{frame}{And Ghidra...}
  \begin{itemize}
    \item Free and open source - Apache 2.0 Licensed or Public Domain
      (choice of contributor)
      \url{https://github.com/NationalSecurityAgency/ghidra}
    \item Has a lot better support for people working on teams then
      IDA 
    \item Security professionals are saying it rivals the
      functionality of IDA
  \end{itemize}
\end{frame}

% Commands to include a figure:
%\begin{figure}
%\includegraphics[width=\textwidth]{your-figure's-file-name}
%\caption{\label{fig:your-figure}Caption goes here.}
%\end{figure}

\section{Theory}

\begin{frame}{What are the Goals of a compiler?}
  \begin{itemize}
    \item Take a programming language that is human-readable and
      writable, into something that the computer can run.
    \item Generally a program is considered compiled if it is in
      assembly code - assembly to machine code is done by a
      \textit{assembler}, not a compiler.
    \item Three parts: parsing, type checking, and code generation [1]
  \end{itemize}
\end{frame}

\begin{frame}{Wait, how do decompilers work anyways? Or even a
    compiler?}
  \begin{itemize}
    \item Option A: Compile down into another
      language to be run. For example, the Python programming language
      is writen in C. Programs that do this translation are often
      called 'compiler compilers'.
    \item Option B (generally better): Compiler the language in it's
      own language, for example: Lisp, C, 
  \end{itemize}

\end{frame}

\end{document}

% [1] http://steve-yegge.blogspot.com/2007/06/rich-programmer-food.html
